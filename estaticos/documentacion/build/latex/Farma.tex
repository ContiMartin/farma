% Generated by Sphinx.
\def\sphinxdocclass{report}
\documentclass[letterpaper,10pt,spanish]{sphinxmanual}

\usepackage[utf8]{inputenc}
\ifdefined\DeclareUnicodeCharacter
  \DeclareUnicodeCharacter{00A0}{\nobreakspace}
\else\fi
\usepackage{cmap}
\usepackage[T1]{fontenc}
\usepackage{amsmath,amssymb}
\usepackage{babel}
\usepackage{times}
\usepackage[Sonny]{fncychap}
\usepackage{longtable}
\usepackage{sphinx}
\usepackage{multirow}
\usepackage{eqparbox}


\addto\captionsspanish{\renewcommand{\figurename}{Figura }}
\addto\captionsspanish{\renewcommand{\tablename}{Tabla }}
\SetupFloatingEnvironment{literal-block}{name=Lista }

\addto\extrasspanish{\def\pageautorefname{página}}

\setcounter{tocdepth}{1}


\title{Farma Documentation}
\date{may. 03, 2016}
\release{}
\author{Murua, Federico - Castro, Lucas Matias - Vergara, Marcos}
\newcommand{\sphinxlogo}{}
\renewcommand{\releasename}{Publicación}
\makeindex

\makeatletter
\def\PYG@reset{\let\PYG@it=\relax \let\PYG@bf=\relax%
    \let\PYG@ul=\relax \let\PYG@tc=\relax%
    \let\PYG@bc=\relax \let\PYG@ff=\relax}
\def\PYG@tok#1{\csname PYG@tok@#1\endcsname}
\def\PYG@toks#1+{\ifx\relax#1\empty\else%
    \PYG@tok{#1}\expandafter\PYG@toks\fi}
\def\PYG@do#1{\PYG@bc{\PYG@tc{\PYG@ul{%
    \PYG@it{\PYG@bf{\PYG@ff{#1}}}}}}}
\def\PYG#1#2{\PYG@reset\PYG@toks#1+\relax+\PYG@do{#2}}

\expandafter\def\csname PYG@tok@gd\endcsname{\def\PYG@tc##1{\textcolor[rgb]{0.63,0.00,0.00}{##1}}}
\expandafter\def\csname PYG@tok@gu\endcsname{\let\PYG@bf=\textbf\def\PYG@tc##1{\textcolor[rgb]{0.50,0.00,0.50}{##1}}}
\expandafter\def\csname PYG@tok@gt\endcsname{\def\PYG@tc##1{\textcolor[rgb]{0.00,0.27,0.87}{##1}}}
\expandafter\def\csname PYG@tok@gs\endcsname{\let\PYG@bf=\textbf}
\expandafter\def\csname PYG@tok@gr\endcsname{\def\PYG@tc##1{\textcolor[rgb]{1.00,0.00,0.00}{##1}}}
\expandafter\def\csname PYG@tok@cm\endcsname{\let\PYG@it=\textit\def\PYG@tc##1{\textcolor[rgb]{0.25,0.50,0.56}{##1}}}
\expandafter\def\csname PYG@tok@vg\endcsname{\def\PYG@tc##1{\textcolor[rgb]{0.73,0.38,0.84}{##1}}}
\expandafter\def\csname PYG@tok@vi\endcsname{\def\PYG@tc##1{\textcolor[rgb]{0.73,0.38,0.84}{##1}}}
\expandafter\def\csname PYG@tok@mh\endcsname{\def\PYG@tc##1{\textcolor[rgb]{0.13,0.50,0.31}{##1}}}
\expandafter\def\csname PYG@tok@cs\endcsname{\def\PYG@tc##1{\textcolor[rgb]{0.25,0.50,0.56}{##1}}\def\PYG@bc##1{\setlength{\fboxsep}{0pt}\colorbox[rgb]{1.00,0.94,0.94}{\strut ##1}}}
\expandafter\def\csname PYG@tok@ge\endcsname{\let\PYG@it=\textit}
\expandafter\def\csname PYG@tok@vc\endcsname{\def\PYG@tc##1{\textcolor[rgb]{0.73,0.38,0.84}{##1}}}
\expandafter\def\csname PYG@tok@il\endcsname{\def\PYG@tc##1{\textcolor[rgb]{0.13,0.50,0.31}{##1}}}
\expandafter\def\csname PYG@tok@go\endcsname{\def\PYG@tc##1{\textcolor[rgb]{0.20,0.20,0.20}{##1}}}
\expandafter\def\csname PYG@tok@cp\endcsname{\def\PYG@tc##1{\textcolor[rgb]{0.00,0.44,0.13}{##1}}}
\expandafter\def\csname PYG@tok@gi\endcsname{\def\PYG@tc##1{\textcolor[rgb]{0.00,0.63,0.00}{##1}}}
\expandafter\def\csname PYG@tok@gh\endcsname{\let\PYG@bf=\textbf\def\PYG@tc##1{\textcolor[rgb]{0.00,0.00,0.50}{##1}}}
\expandafter\def\csname PYG@tok@ni\endcsname{\let\PYG@bf=\textbf\def\PYG@tc##1{\textcolor[rgb]{0.84,0.33,0.22}{##1}}}
\expandafter\def\csname PYG@tok@nl\endcsname{\let\PYG@bf=\textbf\def\PYG@tc##1{\textcolor[rgb]{0.00,0.13,0.44}{##1}}}
\expandafter\def\csname PYG@tok@nn\endcsname{\let\PYG@bf=\textbf\def\PYG@tc##1{\textcolor[rgb]{0.05,0.52,0.71}{##1}}}
\expandafter\def\csname PYG@tok@no\endcsname{\def\PYG@tc##1{\textcolor[rgb]{0.38,0.68,0.84}{##1}}}
\expandafter\def\csname PYG@tok@na\endcsname{\def\PYG@tc##1{\textcolor[rgb]{0.25,0.44,0.63}{##1}}}
\expandafter\def\csname PYG@tok@nb\endcsname{\def\PYG@tc##1{\textcolor[rgb]{0.00,0.44,0.13}{##1}}}
\expandafter\def\csname PYG@tok@nc\endcsname{\let\PYG@bf=\textbf\def\PYG@tc##1{\textcolor[rgb]{0.05,0.52,0.71}{##1}}}
\expandafter\def\csname PYG@tok@nd\endcsname{\let\PYG@bf=\textbf\def\PYG@tc##1{\textcolor[rgb]{0.33,0.33,0.33}{##1}}}
\expandafter\def\csname PYG@tok@ne\endcsname{\def\PYG@tc##1{\textcolor[rgb]{0.00,0.44,0.13}{##1}}}
\expandafter\def\csname PYG@tok@nf\endcsname{\def\PYG@tc##1{\textcolor[rgb]{0.02,0.16,0.49}{##1}}}
\expandafter\def\csname PYG@tok@si\endcsname{\let\PYG@it=\textit\def\PYG@tc##1{\textcolor[rgb]{0.44,0.63,0.82}{##1}}}
\expandafter\def\csname PYG@tok@s2\endcsname{\def\PYG@tc##1{\textcolor[rgb]{0.25,0.44,0.63}{##1}}}
\expandafter\def\csname PYG@tok@nt\endcsname{\let\PYG@bf=\textbf\def\PYG@tc##1{\textcolor[rgb]{0.02,0.16,0.45}{##1}}}
\expandafter\def\csname PYG@tok@nv\endcsname{\def\PYG@tc##1{\textcolor[rgb]{0.73,0.38,0.84}{##1}}}
\expandafter\def\csname PYG@tok@s1\endcsname{\def\PYG@tc##1{\textcolor[rgb]{0.25,0.44,0.63}{##1}}}
\expandafter\def\csname PYG@tok@ch\endcsname{\let\PYG@it=\textit\def\PYG@tc##1{\textcolor[rgb]{0.25,0.50,0.56}{##1}}}
\expandafter\def\csname PYG@tok@m\endcsname{\def\PYG@tc##1{\textcolor[rgb]{0.13,0.50,0.31}{##1}}}
\expandafter\def\csname PYG@tok@gp\endcsname{\let\PYG@bf=\textbf\def\PYG@tc##1{\textcolor[rgb]{0.78,0.36,0.04}{##1}}}
\expandafter\def\csname PYG@tok@sh\endcsname{\def\PYG@tc##1{\textcolor[rgb]{0.25,0.44,0.63}{##1}}}
\expandafter\def\csname PYG@tok@ow\endcsname{\let\PYG@bf=\textbf\def\PYG@tc##1{\textcolor[rgb]{0.00,0.44,0.13}{##1}}}
\expandafter\def\csname PYG@tok@sx\endcsname{\def\PYG@tc##1{\textcolor[rgb]{0.78,0.36,0.04}{##1}}}
\expandafter\def\csname PYG@tok@bp\endcsname{\def\PYG@tc##1{\textcolor[rgb]{0.00,0.44,0.13}{##1}}}
\expandafter\def\csname PYG@tok@c1\endcsname{\let\PYG@it=\textit\def\PYG@tc##1{\textcolor[rgb]{0.25,0.50,0.56}{##1}}}
\expandafter\def\csname PYG@tok@o\endcsname{\def\PYG@tc##1{\textcolor[rgb]{0.40,0.40,0.40}{##1}}}
\expandafter\def\csname PYG@tok@kc\endcsname{\let\PYG@bf=\textbf\def\PYG@tc##1{\textcolor[rgb]{0.00,0.44,0.13}{##1}}}
\expandafter\def\csname PYG@tok@c\endcsname{\let\PYG@it=\textit\def\PYG@tc##1{\textcolor[rgb]{0.25,0.50,0.56}{##1}}}
\expandafter\def\csname PYG@tok@mf\endcsname{\def\PYG@tc##1{\textcolor[rgb]{0.13,0.50,0.31}{##1}}}
\expandafter\def\csname PYG@tok@err\endcsname{\def\PYG@bc##1{\setlength{\fboxsep}{0pt}\fcolorbox[rgb]{1.00,0.00,0.00}{1,1,1}{\strut ##1}}}
\expandafter\def\csname PYG@tok@mb\endcsname{\def\PYG@tc##1{\textcolor[rgb]{0.13,0.50,0.31}{##1}}}
\expandafter\def\csname PYG@tok@ss\endcsname{\def\PYG@tc##1{\textcolor[rgb]{0.32,0.47,0.09}{##1}}}
\expandafter\def\csname PYG@tok@sr\endcsname{\def\PYG@tc##1{\textcolor[rgb]{0.14,0.33,0.53}{##1}}}
\expandafter\def\csname PYG@tok@mo\endcsname{\def\PYG@tc##1{\textcolor[rgb]{0.13,0.50,0.31}{##1}}}
\expandafter\def\csname PYG@tok@kd\endcsname{\let\PYG@bf=\textbf\def\PYG@tc##1{\textcolor[rgb]{0.00,0.44,0.13}{##1}}}
\expandafter\def\csname PYG@tok@mi\endcsname{\def\PYG@tc##1{\textcolor[rgb]{0.13,0.50,0.31}{##1}}}
\expandafter\def\csname PYG@tok@kn\endcsname{\let\PYG@bf=\textbf\def\PYG@tc##1{\textcolor[rgb]{0.00,0.44,0.13}{##1}}}
\expandafter\def\csname PYG@tok@cpf\endcsname{\let\PYG@it=\textit\def\PYG@tc##1{\textcolor[rgb]{0.25,0.50,0.56}{##1}}}
\expandafter\def\csname PYG@tok@kr\endcsname{\let\PYG@bf=\textbf\def\PYG@tc##1{\textcolor[rgb]{0.00,0.44,0.13}{##1}}}
\expandafter\def\csname PYG@tok@s\endcsname{\def\PYG@tc##1{\textcolor[rgb]{0.25,0.44,0.63}{##1}}}
\expandafter\def\csname PYG@tok@kp\endcsname{\def\PYG@tc##1{\textcolor[rgb]{0.00,0.44,0.13}{##1}}}
\expandafter\def\csname PYG@tok@w\endcsname{\def\PYG@tc##1{\textcolor[rgb]{0.73,0.73,0.73}{##1}}}
\expandafter\def\csname PYG@tok@kt\endcsname{\def\PYG@tc##1{\textcolor[rgb]{0.56,0.13,0.00}{##1}}}
\expandafter\def\csname PYG@tok@sc\endcsname{\def\PYG@tc##1{\textcolor[rgb]{0.25,0.44,0.63}{##1}}}
\expandafter\def\csname PYG@tok@sb\endcsname{\def\PYG@tc##1{\textcolor[rgb]{0.25,0.44,0.63}{##1}}}
\expandafter\def\csname PYG@tok@k\endcsname{\let\PYG@bf=\textbf\def\PYG@tc##1{\textcolor[rgb]{0.00,0.44,0.13}{##1}}}
\expandafter\def\csname PYG@tok@se\endcsname{\let\PYG@bf=\textbf\def\PYG@tc##1{\textcolor[rgb]{0.25,0.44,0.63}{##1}}}
\expandafter\def\csname PYG@tok@sd\endcsname{\let\PYG@it=\textit\def\PYG@tc##1{\textcolor[rgb]{0.25,0.44,0.63}{##1}}}

\def\PYGZbs{\char`\\}
\def\PYGZus{\char`\_}
\def\PYGZob{\char`\{}
\def\PYGZcb{\char`\}}
\def\PYGZca{\char`\^}
\def\PYGZam{\char`\&}
\def\PYGZlt{\char`\<}
\def\PYGZgt{\char`\>}
\def\PYGZsh{\char`\#}
\def\PYGZpc{\char`\%}
\def\PYGZdl{\char`\$}
\def\PYGZhy{\char`\-}
\def\PYGZsq{\char`\'}
\def\PYGZdq{\char`\"}
\def\PYGZti{\char`\~}
% for compatibility with earlier versions
\def\PYGZat{@}
\def\PYGZlb{[}
\def\PYGZrb{]}
\makeatother

\renewcommand\PYGZsq{\textquotesingle}

\begin{document}
\shorthandoff{"}
\maketitle
\tableofcontents
\phantomsection\label{index::doc}


Contenidos:


\chapter{Medicamentos}
\label{medicamentos::doc}\label{medicamentos:medicamentos}\label{medicamentos:grupo-farma}
Luego de hacer click en el ítem “Medicamentos” del menú principal el sistema muestra un submenú donde el usuario puede seleccionar la actividad que desea realizar:
\begin{itemize}
\item {} 
Medicamentos

\item {} 
Monodrogas

\item {} 
Presentaciones

\item {} 
Nombres Fantasía

\end{itemize}

Contenidos:


\section{Medicamentos}
\label{medicams::doc}\label{medicams:medicamentos}
Se presentará una pantalla que contendrá un listado con todos los medicamentos que se encuentren registrados en el sistema hasta la fecha. Junto con el listado, se presentarán un conjunto de funcionalidades que permitirá manipular estos medicamentos.

\includegraphics{{medicamentos}.png}

Estas funcionalidades son:
\begin{itemize}
\item {} 
Alta de Medicamento

\item {} 
Modificar Stock Mínimo de Medicamento

\item {} 
Modificar Precio de Venta de Medicamento

\item {} 
Ver Lotes de Medicamento

\item {} 
Eliminar Medicamento

\item {} 
Formulario de Búsqueda

\end{itemize}


\subsection{Alta de Medicamento}
\label{medicams:alta-de-medicamento}
Si el usuario desea agregar un medicamento, deberá presionar el botón de “Alta”.

\includegraphics{{botonalta}.png}

Una vez realizado el paso, el sistema lo redirigirá a la siguiente pantalla:

CAPTURA

En esta parte el usuario se le presentará un formulario y deberá ingresar la información solicitada para dar de alta un nuevo medicamento.
El sistema siempre validará que la información ingresada sea correcta. En caso de que los datos ingresados sean incorrectos el sistema lo informará.
En este punto, las posibles causas de errores son:
\begin{itemize}
\item {} 
Uno o más campos vacíos.

\item {} 
El código de barras del medicamento ya existe.

\item {} 
La monodroga ingresada no existe.

\end{itemize}

Una vez completado el formulario, el usuario tendrá dos opciones: presionar el botón “Guardar y volver” o presionar el botón “Guardar y continuar”.
El botón “Guardar y volver” permite guardar el medicamento en el sistema y volver a la pantalla principal de medicamentos.
El botón “Guardar y continuar” permito guardar el medicamento en el sistema y seguir dando de alta nuevos medicamentos.


\subsection{Modificar Stock Mínimo}
\label{medicams:modificar-stock-minimo}
Si el usuario desea modificar el stock mínimo de un medicamento, deberá seleccionar el botón de “Acción” asociado al medicamento y presionar la pestaña “Modificar Stock Mínimo”.

\includegraphics{{modifstockmin}.png}

Una vez realizado el paso anterior, el sistema lo redirigirá a la siguiente pantalla:

\includegraphics{{modifstockmed}.png}

En esta parte el usuario se le presentará un formulario y deberá actualizar la información del stock asociado al medicamento.
El sistema siempre validará que la información ingresada sea correcta. En caso de que los datos ingresados sean incorrectos el sistema lo informará.
En este punto, las posibles causas de errores son:
\begin{itemize}
\item {} 
No se ingresó un stock mínimo.

\item {} 
El stock mínimo ingresado no posee un formato correcto.

\item {} 
El stock mínimo ingresado es menor a cero.

\end{itemize}

Una vez completado el formulario, el usuario deberá presionar el botón “Guardar Cambios” y el sistema se encargara de actualizar el stock mínimo del medicamento seleccionado.


\subsection{Modificar Precio de Venta}
\label{medicams:modificar-precio-de-venta}
Si el usuario desea modificar el precio de venta de un medicamento, deberá seleccionar el botón de “Acción” asociado al medicamento y presionar la pestaña “Modificar Precio Venta”.

\includegraphics{{modifprecioventa}.png}

Una vez realizado el paso anterior, el sistema lo redirigirá a la siguiente pantalla:

\includegraphics{{modifpreciomed}.png}

En esta parte el usuario se le presentará un formulario y deberá actualizar la información del precio de venta asociado al medicamento.
El sistema siempre validará que la información ingresada sea correcta. En caso de que los datos ingresados sean incorrectos el sistema lo informará.
En este punto, las posibles causas de errores son:
\begin{itemize}
\item {} 
No se ingresó un precio de venta.

\item {} 
El precio de venta ingresado no posee un formato correcto.

\item {} 
El precio de venta ingresado es menor a cero.

\end{itemize}

Una vez completado el formulario, el usuario deberá presionar el botón “Guardar Cambios” y el sistema se encargara de actualizar el precio de venta del medicamento seleccionado.


\subsection{Ver Lotes}
\label{medicams:ver-lotes}
Si el usuario desea ver los lotes de un medicamento, deberá seleccionar el botón de “Acción” asociado al medicamento y presionar la pestaña “Ver Lotes”.

\includegraphics{{verlotes}.png}

Una vez realizado el paso anterior aparecerá la siguiente ventana emergente (modal):

\includegraphics{{lotesmed}.png}

Esta ventana mostrará todos los lotes que estén asociados al medicamento.

En caso de que el medicamento seleccionado no posea lotes activos, es decir que tengan stock, o que no se encuentren dentro del rango de vencimento, el sistema mostrara la siguiente ventana emergente (modal):

\includegraphics{{nolotes}.png}


\subsection{Eliminar Medicamento}
\label{medicams:eliminar-medicamento}
Si el usuario desea eliminar un medicamento, deberá seleccionar el botón de “Acción” asociado al medicamento y presionar la pestaña “Eliminar”.

\includegraphics{{eliminar}.png}

Una vez realizado el paso anterior aparecerá la siguiente ventana emergente (modal):

\includegraphics{{eliminarmed}.png}

En esta parte el usuario deberá decidir si confirma la eliminación del medicamento o no. Si desea confirmar la eliminación deberá presionar el botón “Confirmar”, caso contrario, presionará el botón “Cancelar”.

En caso de que el medicamento posea referencias activas que impidan la eliminacion, el sistema mostrara una ventana emergente (modal) del tipo:

\includegraphics{{noeliminarmed}.png}


\subsection{Formulario de Búsqueda}
\label{medicams:formulario-de-busqueda}
Si el usuario desea visualizar solo aquellos medicamentos que cumplan determinados criterios, deberá utilizar el formulario de búsqueda.

\includegraphics{{busquedamed}.png}

Este formulario cuenta con dos modalidades:
\begin{itemize}
\item {} 
Búsqueda simple: permite buscar los medicamentos por nombre fantasía.

\item {} 
Búsqueda avanzada: permite buscar los medicamentos por nombre fantasía y/o Laboratorio.

\end{itemize}

Todos los campos son opcionales, de no especificarse ningún criterio de búsqueda el sistema mostrará todos los medicamentos.


\section{Monodrogas}
\label{monodrogas:monodrogas}\label{monodrogas::doc}
Se presentará una pantalla que contendrá un listado con todas las monodrogas que se encuentren registrados en el sistema hasta la fecha. Junto con el listado, se presentarán un conjunto de funcionalidades que permitirá manipular estas monodrogas.

CAPTURA

Estas funcionalidades son:
\begin{itemize}
\item {} 
Alta de Monodroga

\item {} 
Modificar Monodroga

\item {} 
Eliminar Monodroga

\item {} 
Formulario de Búsqueda

\end{itemize}


\subsection{Alta de Monodroga}
\label{monodrogas:alta-de-monodroga}
Si el usuario desea agregar una monodroga, deberá presionar el botón de “Alta Monodroga”.

CAPTURA BOTÓN

Una vez realizado el paso, el sistema lo redirigirá a la siguiente pantalla:

CAPTURA

En esta parte el usuario se le presentará un formulario y deberá ingresar la información solicitada para dar de alta una nueva monodroga.
El sistema siempre validará que la información ingresada sea correcta. En caso de que los datos ingresados sean incorrectos el sistema lo informará.
En este punto, las posibles causas de errores son:
\begin{itemize}
\item {} 
No se ingresó un nombre de monodroga.

\item {} 
El nombre de monodroga contiene caracteres especiales o números.

\item {} 
El nombre de la monodroga ya existe en el sistema.

\end{itemize}

Una vez completado el formulario, el usuario tendrá dos opciones: presionar el botón “Guardar y volver” o presionar el botón “Guardar y continuar”.
El botón “Guardar y volver” permite guardar la monodroga en el sistema y volver a la pantalla principal de monodrogas.
El botón “Guardar y continuar” permito guardar la monodroga en el sistema y seguir dando de alta nuevos medicamentos.


\subsection{Modificar Monodroga}
\label{monodrogas:modificar-monodroga}
Si el usuario desea modificar una monodroga, deberá seleccionar el botón de “Acción” asociado a la monodroga y presionar la pestaña “Modificar”.

CAPTURA BOTÓN

Una vez realizado el paso anterior, el sistema lo redirigirá a la siguiente pantalla:

CAPTURA

En esta parte el usuario se le presentará un formulario y deberá actualizar la información asociada a la monodroga.
El sistema siempre validará que la información ingresada sea correcta. En caso de que los datos ingresados sean incorrectos el sistema lo informará.
En este punto, las posibles causas de errores son:
\begin{itemize}
\item {} 
No se ingresó un nombre de monodroga.

\item {} 
El nombre de monodroga contiene caracteres especiales o números.

\item {} 
El nombre de la monodroga ya existe en el sistema.

\end{itemize}

Una vez completado el formulario, el usuario deberá presionar el botón “Guardar Cambios” y el sistema se encargara de actualizar la información de la monodroga seleccionada.


\subsection{Eliminar Monodroga}
\label{monodrogas:eliminar-monodroga}
Si el usuario desea eliminar una monodroga, deberá seleccionar el botón de “Acción” asociado a la monodroga y presionar la pestaña “Eliminar”.

CAPTURA BOTÓN

Una vez realizado el paso anterior aparecerá la siguiente ventana emergente (modal):

CAPTURA

En esta parte el usuario deberá decidir si confirma la eliminación de la monodroga o no. Si desea confirmar la eliminación deberá presionar el botón “Confirmar”, caso contrario, presionará el botón “Cancelar”.


\subsection{Formulario de Búsqueda}
\label{monodrogas:formulario-de-busqueda}
Si el usuario desea visualizar solo aquellas monodrogas que cumplan determinados criterios, deberá utilizar el formulario de búsqueda.

CAPTURA

Este formulario sólo cuenta con la opción de buscar por el nombre de la monodroga.

Este campo es opcional y de no especificarse ningún criterio de búsqueda el sistema mostrará todas las monodrogas.


\section{Presentaciones}
\label{presentaciones::doc}\label{presentaciones:presentaciones}
Se presentará una pantalla que contendrá un listado con todas las Presentaciones que se encuentren registradas en el sistema hasta la fecha. Junto con el listado, se presentarán un conjunto de funcionalidades que permitirá manipular estas Presentaciones.

CAPTURA

Estas funcionalidades son:
\begin{itemize}
\item {} 
Alta de Presentación

\item {} 
Modificar Presentación

\item {} 
Eliminar Presentación

\item {} 
Formulario de Búsqueda

\end{itemize}


\subsection{Alta de Presentación}
\label{presentaciones:alta-de-presentacion}
Si el usuario desea agregar una presentación, deberá presionar el botón de “Alta Presentación”.

CAPTURA BOTÓN

Una vez realizado el paso, el sistema lo redirigirá a la siguiente pantalla:

CAPTURA

En esta parte el usuario se le presentará un formulario y deberá ingresar la información solicitada para dar de alta una nueva presentación.
El sistema siempre validará que la información ingresada sea correcta. En caso de que los datos ingresados sean incorrectos el sistema lo informará.
En este punto, las posibles causas de errores son:
\begin{itemize}
\item {} 
Uno o más campos vacios.

\item {} 
La cantidad ingresada no posee el formato correcto.

\item {} 
La cantidad ingresada es menor a cero.

\end{itemize}

Una vez completado el formulario, el usuario tendrá dos opciones: presionar el botón “Guardar y volver” o presionar el botón “Guardar y continuar”.
El botón “Guardar y volver” permite guardar la presentación en el sistema y volver a la pantalla principal de Presentaciones.
El botón “Guardar y continuar” permito guardar la presentación en el sistema y seguir dando de alta nuevos medicamentos.


\subsection{Modificar Presentación}
\label{presentaciones:modificar-presentacion}
Si el usuario desea modificar una presentación, deberá seleccionar el botón de “Acción” asociado a la presentación y presionar la pestaña “Modificar”.

CAPTURA BOTÓN

Una vez realizado el paso anterior, el sistema lo redirigirá a la siguiente pantalla:

CAPTURA

En esta parte el usuario se le presentará un formulario y deberá actualizar la información asociada a la presentación.
El sistema siempre validará que la información ingresada sea correcta. En caso de que los datos ingresados sean incorrectos el sistema lo informará.
En este punto, las posibles causas de errores son:
\begin{itemize}
\item {} 
Uno o más campos vacios.

\item {} 
La cantidad ingresada no posee el formato correcto.

\item {} 
La cantidad ingresada es menor a cero.

\end{itemize}

Una vez completado el formulario, el usuario deberá presionar el botón “Guardar Cambios” y el sistema se encargara de actualizar la información de la presentación seleccionada.


\subsection{Eliminar Presentación}
\label{presentaciones:eliminar-presentacion}
Si el usuario desea eliminar una presentación, deberá seleccionar el botón de “Acción” asociado a la presentación y presionar la pestaña “Eliminar”.

CAPTURA BOTÓN

Una vez realizado el paso anterior aparecerá la siguiente ventana emergente (modal):

CAPTURA

En esta parte el usuario deberá decidir si confirma la eliminación de la presentación o no. Si desea confirmar la eliminación deberá presionar el botón “Confirmar”, caso contrario, presionará el botón “Cancelar”.


\subsection{Formulario de Búsqueda}
\label{presentaciones:formulario-de-busqueda}
Si el usuario desea visualizar solo aquellas Presentaciones que cumplan determinados criterios, deberá utilizar el formulario de búsqueda.

CAPTURA

Este formulario sólo cuenta con la opción de buscar por la descripción de la Monodroga.

Este campo es opcional y de no especificarse ningún criterio de búsqueda el sistema mostrará todas las Presentaciones.


\section{Nombres Fantasia}
\label{nombresfantasia:nombres-fantasia}\label{nombresfantasia::doc}
Se presentará una pantalla que contendrá un listado con todos los nombres fantasía que se encuentren registradas en el sistema hasta la fecha. Junto con el listado, se presentarán un conjunto de funcionalidades que permitirá manipular estas Presentaciones.

CAPTURA

Estas funcionalidades son:
\begin{itemize}
\item {} 
Alta de Nombre Fantasía

\item {} 
Modificar Nombre Fantasía

\item {} 
Eliminar Nombre Fantasía

\item {} 
Formulario de Búsqueda

\end{itemize}


\subsection{Alta de Nombre Fantasía}
\label{nombresfantasia:alta-de-nombre-fantasia}
Si el usuario desea agregar un nombre fantasía, deberá presionar el botón de “Alta Nombre Fantasía”.

CAPTURA BOTÓN

Una vez realizado el paso, el sistema lo redirigirá a la siguiente pantalla:

CAPTURA

En esta parte el usuario se le presentará un formulario y deberá ingresar la información solicitada para dar de alta un nuevo nombre fantasía.
El sistema siempre validará que la información ingresada sea correcta. En caso de que los datos ingresados sean incorrectos el sistema lo informará.
En este punto, las posibles causas de errores son:
\begin{itemize}
\item {} 
No se ingresó un nombre fantasía.

\item {} 
El nombre fantasía ingresado ya existe en el sistema.

\end{itemize}

Una vez completado el formulario, el usuario tendrá dos opciones: presionar el botón “Guardar y volver” o presionar el botón “Guardar y continuar”.
El botón “Guardar y volver” permite guardar el nombre fantasía en el sistema y volver a la pantalla principal de nombres fantasía.
El botón “Guardar y continuar” permito guardar el nombre fantasía en el sistema y seguir dando de alta nuevos nombres fantasía.


\subsection{Modificar Nombre Fantasía}
\label{nombresfantasia:modificar-nombre-fantasia}
Si el usuario desea modificar un nombre fantasía, deberá seleccionar el botón de “Acción” asociado al nombre fantasía y presionar la pestaña “Modificar”.

CAPTURA BOTÓN

Una vez realizado el paso anterior, el sistema lo redirigirá a la siguiente pantalla:

CAPTURA

En esta parte el usuario se le presentará un formulario y deberá actualizar la información asociada al nombre fantasía.
El sistema siempre validará que la información ingresada sea correcta. En caso de que los datos ingresados sean incorrectos el sistema lo informará.
En este punto, las posibles causas de errores son:
\begin{itemize}
\item {} 
No se ingresó un nombre fantasía.

\item {} 
El nombre fantasía ingresado ya existe en el sistema.

\end{itemize}

Una vez completado el formulario, el usuario deberá presionar el botón “Guardar Cambios” y el sistema se encargara de actualizar la información del nombre fantasía seleccionado.


\subsection{Eliminar Nombre Fantasía}
\label{nombresfantasia:eliminar-nombre-fantasia}
Si el usuario desea eliminar un nombre fantasía, deberá seleccionar el botón de “Acción” asociado al nombre fantasía y presionar la pestaña “Eliminar”.

CAPTURA BOTÓN

Una vez realizado el paso anterior aparecerá la siguiente ventana emergente (modal):

CAPTURA

En esta parte el usuario deberá decidir si confirma la eliminación del nombre fantasía o no. Si desea confirmar la eliminación deberá presionar el botón “Confirmar”, caso contrario, presionará el botón “Cancelar”.


\subsection{Formulario de Búsqueda}
\label{nombresfantasia:formulario-de-busqueda}
Si el usuario desea visualizar solo aquellos nombres fantasía que cumplan determinados criterios, deberá utilizar el formulario de búsqueda.

CAPTURA

Este formulario sólo cuenta con la opción de buscar por el campo nombre del nombre fantasía.

Este campo es opcional y de no especificarse ningún criterio de búsqueda el sistema mostrará todos los nombres fantasía.


\chapter{Organizaciones}
\label{organizaciones::doc}\label{organizaciones:organizaciones}
Luego de hacer “click” en  la leyenda ``Organizaciones'', el sistema muestra un submenú donde el usuario puede seleccionar la organizacion sobre la cual realizar acciones.

Contenido:


\section{Farmacias}
\label{farmacias:farmacias}\label{farmacias::doc}
Se presentará una pantalla que contendrá un listado con todas las farmacias que se encuentren registradas en el sistema hasta la fecha. Junto con el listado, se presentarán un conjunto de funcionalidades que permitirán manipular estas farmacias.

CAPTURA

Estas funcionalidades son:
\begin{itemize}
\item {} 
Alta de Farmacia

\item {} 
Baja de Farmacia

\item {} 
Modificacion de Farmacia

\item {} 
Formulario de Búsqueda

\end{itemize}


\subsection{Alta de Farmacia}
\label{farmacias:alta-de-farmacia}
Si el usuario desea crear una nueva farmacia, deberá presionar el botón “Alta”. Una vez presionado este botón el sistema lo redirigirá a la siguiente pantalla.

CAPTURA

En este punto el usuario deberá ingresar los datos de la nueva farmacia. Estos datos son:

Campos Obligatorios:

\begin{Verbatim}[commandchars=\\\{\}]
\PYG{o}{\PYGZhy{}} \PYG{n}{Razón} \PYG{n}{social}
\PYG{o}{\PYGZhy{}} \PYG{n}{Cuit}
\PYG{o}{\PYGZhy{}} \PYG{n}{Localidad}
\PYG{o}{\PYGZhy{}} \PYG{n}{Dirección}
\end{Verbatim}

Campos opcionales:

\begin{Verbatim}[commandchars=\\\{\}]
\PYG{o}{\PYGZhy{}} \PYG{n}{Nombre} \PYG{n}{de} \PYG{n}{encargado}
\PYG{o}{\PYGZhy{}} \PYG{n}{Teléfono}
\PYG{o}{\PYGZhy{}} \PYG{n}{Email}
\end{Verbatim}

Luego de ingresar todos los datos, el usuario podra confirmar su grabación. Para esto cuenta con los botones “Guardar y volver” que redirige al listado inicial de una organización, y “Guardar y continuar” que mantiene la pantalla activa para crear una nueva farmacia.

El sistema siempre validará que la información ingresada sea correcta. En caso de que los datos ingresados sean incorrectos el sistema lo informará.
En este punto, las posibles causas de errores son:
\begin{itemize}
\item {} 
No se ingreso una razon social.

\item {} 
La razon social ingresada no posee un formato correcto.

\item {} 
No se ingreso un CUIT.

\item {} 
El CUIT ingresado no posee un formato correcto.

\item {} 
No se ingreso una localidad.

\item {} 
La localidad ingresada no posee un formato correcto.

\item {} 
No se ingreso una direccion.

\item {} 
La direccion ingresada no posee un formato correcto.

\item {} 
El nombre de encargado ingresado no posee un formato correcto.

\item {} 
El telefono ingresado no posee un formato correcto.

\item {} 
El email ingresado no posee un formato correcto.

\end{itemize}


\subsection{Baja de Farmacia}
\label{farmacias:baja-de-farmacia}
Si el usuario desea eliminar una farmacia, deberá hacer “click” en la fila correspondiente y presionar el botón de “Acción” y seleccionar la opción eliminar.

CAPTURA BOTÓN

Una vez realizado el paso anterior aparecerá la siguiente ventana emergente (modal):

CAPTURA

En esta parte el usuario deberá decidir si confirma la eliminación de la farmacia. Si desea confirmar la eliminación deberá presionar el botón “Confirmar”, caso contrario, presionará el botón “Cancelar”.


\subsection{Modificacion de Farmacia}
\label{farmacias:modificacion-de-farmacia}
Si el usuario desea modificar una farmacia, deberá hacer “click” en la fila correspondiente y presionar el botón de “Acción” y seleccionar la opción modificar.
Una vez presionado este botón el sistema lo redirigirá a la siguiente pantalla.

CAPTURA

En esta parte el usuario se le presentará un formulario con la información modificable de la farmacia, y podra actualizar la información que considere necesaria.

Una vez modificado el formulario, el usuario deberá presionar el botón “Guardar cambios” y el sistema se encargara de actualizar la información de la farmacia seleccionada.

El sistema siempre validará que la información ingresada sea correcta. En caso de que los datos ingresados sean incorrectos el sistema lo informará.
En este punto, las posibles causas de errores son:
\begin{itemize}
\item {} 
No se ingreso una localidad.

\item {} 
La localidad ingresada no posee un formato correcto.

\item {} 
No se ingreso una direccion.

\item {} 
La direccion ingresada no posee un formato correcto.

\item {} 
El nombre de encargado ingresado no posee un formato correcto.

\item {} 
El telefono ingresado no posee un formato correcto.

\item {} 
El email ingresado no posee un formato correcto.

\end{itemize}


\subsection{Formulario de Búsqueda}
\label{farmacias:formulario-de-busqueda}
Si el usuario desea visualizar solo aquellas farmacias que cumplan determinados criterios, deberá utilizar el formulario de búsqueda.

CAPTURA

Este formulario cuenta con dos modalidades:
\begin{itemize}
\item {} 
Búsqueda simple: permite buscar las farmacias por razon social.

\item {} 
Búsqueda avanzada: permite buscar las farmacias por razon social y/o localidad.

\end{itemize}

Todos los campos son opcionales, de no especificarse ningún criterio de búsqueda el sistema mostrará todos los pedidos de farmacia.


\section{Clínicas}
\label{clinicas:clinicas}\label{clinicas::doc}
Se presentará una pantalla que contendrá un listado con todas las clínicas que se encuentren registradas en el sistema hasta la fecha. Junto con el listado, se presentarán un conjunto de funcionalidades que permitirán manipular estas clínicas.

CAPTURA

Estas funcionalidades son:
\begin{itemize}
\item {} 
Alta de Clínica

\item {} 
Baja de Clínica

\item {} 
Modificacion de Clínica

\item {} 
Formulario de Búsqueda

\end{itemize}


\subsection{Alta de Clínica}
\label{clinicas:alta-de-clinica}
Si el usuario desea crear una nueva clínica, deberá presionar el botón “Alta”. Una vez presionado este botón el sistema lo redirigirá a la siguiente pantalla.

CAPTURA

En este punto el usuario deberá ingresar los datos de la nueva clínica. Estos datos son:

Campos Obligatorios:

\begin{Verbatim}[commandchars=\\\{\}]
\PYG{o}{\PYGZhy{}} \PYG{n}{Razón} \PYG{n}{social}
\PYG{o}{\PYGZhy{}} \PYG{n}{Cuit}
\PYG{o}{\PYGZhy{}} \PYG{n}{Localidad}
\PYG{o}{\PYGZhy{}} \PYG{n}{Dirección}
\PYG{o}{\PYGZhy{}} \PYG{n}{Obra} \PYG{n}{social}
\end{Verbatim}

Campos opcionales:

\begin{Verbatim}[commandchars=\\\{\}]
\PYG{o}{\PYGZhy{}} \PYG{n}{Teléfono}
\PYG{o}{\PYGZhy{}} \PYG{n}{Email}
\end{Verbatim}

Luego de ingresar todos los datos, el usuario podra confirmar su grabación. Para esto cuenta con los botones “Guardar y volver” que redirige al listado inicial de una organización, y “Guardar y continuar” que mantiene la pantalla activa para crear una nueva clínica.

El sistema siempre validará que la información ingresada sea correcta. En caso de que los datos ingresados sean incorrectos el sistema lo informará.
En este punto, las posibles causas de errores son:
\begin{itemize}
\item {} 
No se ingreso una razon social.

\item {} 
La razon social ingresada no posee un formato correcto.

\item {} 
No se ingreso un CUIT.

\item {} 
El CUIT ingresado no posee un formato correcto.

\item {} 
No se ingreso una localidad.

\item {} 
La localidad ingresada no posee un formato correcto.

\item {} 
No se ingreso una direccion.

\item {} 
La direccion ingresada no posee un formato correcto.

\item {} 
No se ingreso una obra social.

\item {} 
La obra social ingresada no posee un formato correcto.

\item {} 
El telefono ingresado no posee un formato correcto.

\item {} 
El email ingresado no posee un formato correcto.

\end{itemize}


\subsection{Baja de Clínica}
\label{clinicas:baja-de-clinica}
Si el usuario desea eliminar una clínica, deberá hacer “click” en la fila correspondiente y presionar el botón de “Acción” y seleccionar la opción eliminar.

CAPTURA BOTÓN

Una vez realizado el paso anterior aparecerá la siguiente ventana emergente (modal):

CAPTURA

En esta parte el usuario deberá decidir si confirma la eliminación de la clínica. Si desea confirmar la eliminación deberá presionar el botón “Confirmar”, caso contrario, presionará el botón “Cancelar”.


\subsection{Modificacion de Clínica}
\label{clinicas:modificacion-de-clinica}
Si el usuario desea modificar una clínica, deberá hacer “click” en la fila correspondiente y presionar el botón de “Acción” y seleccionar la opción modificar.
Una vez presionado este botón el sistema lo redirigirá a la siguiente pantalla.

CAPTURA

En esta parte el usuario se le presentará un formulario con la información modificable de la clínica, y podra actualizar la información que considere necesaria.

Una vez modificado el formulario, el usuario deberá presionar el botón “Guardar cambios” y el sistema se encargara de actualizar la información de la clínica seleccionada.

El sistema siempre validará que la información ingresada sea correcta. En caso de que los datos ingresados sean incorrectos el sistema lo informará.
En este punto, las posibles causas de errores son:
\begin{itemize}
\item {} 
No se ingreso una localidad.

\item {} 
La localidad ingresada no posee un formato correcto.

\item {} 
No se ingreso una direccion.

\item {} 
La direccion ingresada no posee un formato correcto.

\item {} 
No se ingreso una obra social.

\item {} 
La obra social ingresada no posee un formato correcto.

\item {} 
El telefono ingresado no posee un formato correcto.

\item {} 
El email ingresado no posee un formato correcto.

\end{itemize}


\subsection{Formulario de Búsqueda}
\label{clinicas:formulario-de-busqueda}
Si el usuario desea visualizar solo aquellas clínicas que cumplan determinados criterios, deberá utilizar el formulario de búsqueda.

CAPTURA

Este formulario cuenta con dos modalidades:
\begin{itemize}
\item {} 
Búsqueda simple: permite buscar las clínicas por razon social.

\item {} 
Búsqueda avanzada: permite buscar las clínicas por razon social y/o localidad y/u obra social.

\end{itemize}

Todos los campos son opcionales, de no especificarse ningún criterio de búsqueda el sistema mostrará todos los pedidos de clínica.


\section{Laboratorios}
\label{laboratorios::doc}\label{laboratorios:laboratorios}
Se presentará una pantalla que contendrá un listado con todas las laboratorios que se encuentren registradas en el sistema hasta la fecha. Junto con el listado, se presentarán un conjunto de funcionalidades que permitirán manipular estas laboratorios.

CAPTURA

Estas funcionalidades son:
\begin{itemize}
\item {} 
Alta de Laboratorio

\item {} 
Baja de Laboratorio

\item {} 
Modificacion de Laboratorio

\item {} 
Formulario de Búsqueda

\end{itemize}


\subsection{Alta de Laboratorio}
\label{laboratorios:alta-de-laboratorio}
Si el usuario desea crear una nueva laboratorio, deberá presionar el botón “Alta”. Una vez presionado este botón el sistema lo redirigirá a la siguiente pantalla.

CAPTURA

En este punto el usuario deberá ingresar los datos de la nueva laboratorio. Estos datos son:

Campos Obligatorios:

\begin{Verbatim}[commandchars=\\\{\}]
\PYG{o}{\PYGZhy{}} \PYG{n}{Razón} \PYG{n}{Social}
\PYG{o}{\PYGZhy{}} \PYG{n}{Cuit}
\PYG{o}{\PYGZhy{}} \PYG{n}{Localidad}
\PYG{o}{\PYGZhy{}} \PYG{n}{Dirección}
\end{Verbatim}

Campos opcionales:

\begin{Verbatim}[commandchars=\\\{\}]
\PYG{o}{\PYGZhy{}} \PYG{n}{Teléfono}
\PYG{o}{\PYGZhy{}} \PYG{n}{Email}
\end{Verbatim}

Luego de ingresar todos los datos, el usuario podra confirmar su grabación. Para esto cuenta con los botones “Guardar y volver” que redirige al listado inicial de una organización, y “Guardar y continuar” que mantiene la pantalla activa para crear una nueva laboratorio.

El sistema siempre validará que la información ingresada sea correcta. En caso de que los datos ingresados sean incorrectos el sistema lo informará.
En este punto, las posibles causas de errores son:
\begin{itemize}
\item {} 
No se ingreso una razon social.

\item {} 
La razon social ingresada no posee un formato correcto.

\item {} 
No se ingreso un CUIT.

\item {} 
El CUIT ingresado no posee un formato correcto.

\item {} 
No se ingreso una localidad.

\item {} 
La localidad ingresada no posee un formato correcto.

\item {} 
No se ingreso una direccion.

\item {} 
La direccion ingresada no posee un formato correcto.

\item {} 
El telefono ingresado no posee un formato correcto.

\item {} 
El email ingresado no posee un formato correcto.

\end{itemize}


\subsection{Baja de Laboratorio}
\label{laboratorios:baja-de-laboratorio}
Si el usuario desea eliminar una laboratorio, deberá hacer “click” en la fila correspondiente y presionar el botón de “Acción” y seleccionar la opción eliminar.

CAPTURA BOTÓN

Una vez realizado el paso anterior aparecerá la siguiente ventana emergente (modal):

CAPTURA

En esta parte el usuario deberá decidir si confirma la eliminación de la laboratorio. Si desea confirmar la eliminación deberá presionar el botón “Confirmar”, caso contrario, presionará el botón “Cancelar”.


\subsection{Modificacion de Laboratorio}
\label{laboratorios:modificacion-de-laboratorio}
Si el usuario desea modificar una laboratorio, deberá hacer “click” en la fila correspondiente y presionar el botón de “Acción” y seleccionar la opción modificar.
Una vez presionado este botón el sistema lo redirigirá a la siguiente pantalla.

CAPTURA

En esta parte el usuario se le presentará un formulario con la información modificable de la laboratorio, y podra actualizar la información que considere necesaria.

Una vez modificado el formulario, el usuario deberá presionar el botón “Guardar cambios” y el sistema se encargara de actualizar la información de la laboratorio seleccionada.

El sistema siempre validará que la información ingresada sea correcta. En caso de que los datos ingresados sean incorrectos el sistema lo informará.
En este punto, las posibles causas de errores son:
\begin{itemize}
\item {} 
No se ingreso una localidad.

\item {} 
La localidad ingresada no posee un formato correcto.

\item {} 
No se ingreso una direccion.

\item {} 
La direccion ingresada no posee un formato correcto.

\item {} 
El telefono ingresado no posee un formato correcto.

\item {} 
El email ingresado no posee un formato correcto.

\end{itemize}


\subsection{Formulario de Búsqueda}
\label{laboratorios:formulario-de-busqueda}
Si el usuario desea visualizar solo aquellas laboratorios que cumplan determinados criterios, deberá utilizar el formulario de búsqueda.

CAPTURA

Este formulario cuenta con dos modalidades:
\begin{itemize}
\item {} 
Búsqueda simple: permite buscar las laboratorios por razon social.

\item {} 
Búsqueda avanzada: permite buscar las laboratorios por razon social y/o localidad.

\end{itemize}

Todos los campos son opcionales, de no especificarse ningún criterio de búsqueda el sistema mostrará todos los pedidos de laboratorio.


\section{Obras Sociales}
\label{obras::doc}\label{obras:obras-sociales}

\chapter{Pedidos}
\label{pedidos::doc}\label{pedidos:pedidos}
Luego de hacer “click” en  la leyenda “Pedidos”, el sistema muestra un submenú donde el usuario puede seleccionar la actividad que desea realizar.

Contenido:


\section{Pedidos de Farmacia}
\label{pedidosfarmacia::doc}\label{pedidosfarmacia:pedidos-de-farmacia}
Se presentará una pantalla que contendrá un listado con todos los pedidos de farmacia que se encuentren registrados en el sistema hasta la fecha. Junto con el listado, se presentarán un conjunto de funcionalidades que permitirán manipular estos pedidos.

CAPTURA

Estas funcionalidades son:
\begin{itemize}
\item {} 
Alta de Pedido

\item {} 
Ver detalles del Pedido

\item {} 
Ver Remitos del Pedido

\item {} 
Formulario de Búsqueda

\end{itemize}


\subsection{Alta de Pedido}
\label{pedidosfarmacia:alta-de-pedido}
Si el usuario desea crear un nuevo pedido de farmacia, deberá presionar el botón “Alta”. Una vez presionado este botón el sistema lo redirigirá a la siguiente pantalla

CAPTURA

En este punto el usuario deberá seleccionar la farmacia que solicito el pedido y la correspondiente fecha en que fue solicitado a la empresa; luego presionar el botón “Crear pedido”.

El sistema siempre validará que la información ingresada sea correcta. En caso de que los datos ingresados sean incorrectos el sistema lo informará.
En este punto, las posibles causas de errores son:
\begin{itemize}
\item {} 
La farmacia ingresada no existe.

\item {} 
La fecha no es válida.

\item {} 
La fecha es incorrecta en un sentido “temporal”

\end{itemize}

Una vez presionado el botón “Crear pedido”, se mostrará la siguiente pantalla:

CAPTURA

Esta pantalla es la encargada de visualizar los detalles que se correspondan con el pedido de farmacia.
Esta pantalla ofrece las siguientes funcionalidades para manipular el pedido de farmacia:
\begin{itemize}
\item {} 
Alta Detalle.

\item {} 
Modificación Detalle.

\item {} 
Baja Detalle.

\item {} 
Registrar Pedido.

\end{itemize}


\subsubsection{Alta Detalle}
\label{pedidosfarmacia:alta-detalle}
Si el usuario desea agregar un detalle al pedido de farmacia, deberá presionar el botón de “Alta detalle”.

CAPTURA BOTÓN

Una vez realizado el paso anterior aparecerá la siguiente ventana emergente (modal):

CAPTURA

En esta parte el usuario se le presentará un formulario y deberá ingresar la información solicitada para dar de alta un nuevo detalle.
El sistema siempre validará que la información ingresada sea correcta. En caso de que los datos ingresados sean incorrectos el sistema lo informará.
En este punto, las posibles causas de errores son:
\begin{itemize}
\item {} 
No se seleccionó un medicamento.

\item {} 
No se ingresó una cantidad.

\item {} 
La cantidad ingresada no posee un formato correcto.

\item {} 
La cantidad ingresada es menor a cero.

\end{itemize}

Una vez completado el formulario, el usuario deberá presionar el botón “Guardar” y el sistema se encargara de agregar el nuevo detalle al pedido.
El usuario podrá seguir dando de alta nuevos detalles, hasta donde considere necesario. Una vez que esto suceda deberá presionar el botón “Cerrar” y la ventana emergente desaparecerá.


\subsubsection{Baja Detalle}
\label{pedidosfarmacia:baja-detalle}
Si el usuario desea eliminar un detalle del pedido de farmacia, deberá seleccionar el detalle que desea eliminar y presionar el botón de “Baja detalle”.

CAPTURA BOTÓN

Una vez realizado el paso anterior aparecerá la siguiente ventana emergente (modal):

CAPTURA

En esta parte el usuario deberá decidir si confirma la eliminación del detalle o no. Si desea confirmar la eliminación deberá presionar el botón “Confirmar”, caso contrario, presionará el botón “Cancelar”.


\subsubsection{Modificar Detalle}
\label{pedidosfarmacia:modificar-detalle}
Si el usuario desea modificar un detalle del pedido de farmacia, deberá seleccionar el detalle que desea actualizar y presionar el botón de “Modificar detalle”.

CAPTURA

Una vez realizado el paso anterior aparecerá la siguiente ventana emergente (modal):

CAPTURA

En esta parte el usuario se le presentará un formulario con la información actual del detalle y deberá modificar la información que considere necesaria.
El sistema siempre validará que la información ingresada sea correcta. En caso de que los datos ingresados sean incorrectos el sistema lo informará.
En este punto, las posibles causas de errores son:
\begin{itemize}
\item {} 
No se ingresó una cantidad.

\item {} 
La cantidad ingresada no posee un formato correcto.

\item {} 
La cantidad ingresada es menor a cero.

\end{itemize}

Una vez completado el formulario, el usuario deberá presionar el botón “Guardar” y el sistema se encargara de actualizar la información del detalle seleccionado.


\subsubsection{Registrar Pedido}
\label{pedidosfarmacia:registrar-pedido}
Si el usuario desea registrar el pedido de farmacia, deberá presionar el botón “Registrar”.

CAPTURA

El sistema siempre validará que la información del pedido A de farmacia sea correcta. En caso de que esta información sea incorrecta el sistema lo informará.
En este punto, las posibles causas de errores son:
\begin{itemize}
\item {} 
El pedido no contiene detalles

\item {} 
El pedido ya ha sido registrado anteriormente

\end{itemize}

Una vez presionado el botón “Registrar”, el sistema se encargará de crear el pedido de farmacia y se mostrará la siguiente ventana emergente (modal).

CAPTURA


\subsection{Formulario de Búsqueda}
\label{pedidosfarmacia:formulario-de-busqueda}
Si el usuario desea visualizar solo aquellos pedidos de farmacia que cumplan determinados criterios, deberá utilizar el formulario de búsqueda.

CAPTURA

Este formulario cuenta con dos modalidades:
\begin{itemize}
\item {} 
Búsqueda simple: permite buscar los pedidos de farmacia por farmacia.

\item {} 
Búsqueda avanzada: permite buscar los pedidos de farmacia por farmacia y/o  fecha desde y/o fecha hasta y/o estado del pedido.

\end{itemize}

Todos los campos son opcionales, de no especificarse ningún criterio de búsqueda el sistema mostrará todos los pedidos de farmacia.


\subsection{Ver detalles del Pedido}
\label{pedidosfarmacia:ver-detalles-del-pedido}
Si el usuario desea ver los detalles de un pedido, deberá seleccionar el botón de “Acción” asociado al pedido de farmacia y presionar la pestaña “Ver detalles”.

CAPTURA

Una vez realizado el paso anterior aparecerá la siguiente ventana emergente (modal):

CAPTURA

Esta ventana mostrará todos los detalles que estén asociados al pedido de farmacia.


\subsection{Ver Remitos del Pedido}
\label{pedidosfarmacia:ver-remitos-del-pedido}
Si el usuario desea ver los remitos asociados a un pedido, deberá seleccionar el botón de “Acción” asociado al pedido de farmacia y presionar la pestaña “Ver Remitos”.

CAPTURA

Una vez realizado el paso anterior aparecerá la siguiente ventana emergente (modal):

CAPTURA

Esta ventana mostrará todos los remitos  que estén asociados al pedido de farmacia.

En caso de que el pedido no tenga remitos asociados el sistema lo informará.

Si se desea generar el remito en un pdf, el usuario deberá seleccionar el botón asociado al remito correspondiente y el sistema se encargará de generar el mismo.

CAPTURA


\section{Pedidos de Clínica}
\label{pedidosclinica:pedidos-de-clinica}\label{pedidosclinica::doc}
Se presentará una pantalla que contendrá un listado con todos los pedidos de clínica que se encuentren registrados en el sistema hasta la fecha. Junto con el listado, se presentarán un conjunto de funcionalidades que permitirán manipular estos pedidos.

CAPTURA

Estas funcionalidades son:
\begin{itemize}
\item {} 
Alta de Pedido

\item {} 
Ver detalles del Pedido

\item {} 
Ver Remitos del Pedido

\item {} 
Formulario de Búsqueda

\end{itemize}


\subsection{Alta de Pedido}
\label{pedidosclinica:alta-de-pedido}
Si el usuario desea crear un nuevo pedido de clínica, deberá presionar el botón “Alta”. Una vez presionado este botón  el sistema lo redirigirá a la siguiente pantalla

CAPTURA

En este punto el usuario deberá seleccionar la clínica que solicito el pedido, la obra social con la que trabaja, el médico que audito el pedido y la fecha en que fue solicitado a la empresa. Una vez completa esta información presionar el botón “Crear pedido”.

El sistema siempre validará que la información ingresada sea correcta. En caso de que los datos ingresados sean incorrectos el sistema lo informará.
En este punto, las posibles causas de errores son:
\begin{itemize}
\item {} 
La clínica ingresada no existe.

\item {} 
La fecha no es válida.

\item {} 
La fecha es incorrecta en un sentido “temporal”

\end{itemize}

Una vez presionado el botón “Crear pedido”, se mostrará la siguiente pantalla:

CAPTURA

Esta pantalla es la encargada de visualizar los detalles que se correspondan con el pedido de clínica.
Los detalles del pedido de clínica solo podrán contener medicamentos que se encuentren en stock.

Esta pantalla ofrece las siguientes funcionalidades para manipular el pedido de clínica:
\begin{itemize}
\item {} 
Alta Detalle.

\item {} 
Modificación Detalle.

\item {} 
Baja Detalle.

\item {} 
Registrar Pedido.

\end{itemize}


\subsubsection{Alta Detalle}
\label{pedidosclinica:alta-detalle}
Si el usuario desea agregar un detalle al pedido de clínica, deberá presionar el botón de “Alta detalle”.

CAPTURA BOTÓN

Una vez realizado el paso anterior aparecerá la siguiente ventana emergente (modal):

CAPTURA

En esta parte el usuario se le presentará un formulario y deberá ingresar la información solicitada para dar de alta un nuevo detalle.
El sistema siempre validará que la información ingresada sea correcta. En caso de que los datos ingresados sean incorrectos el sistema lo informará.
En este punto, las posibles causas de errores son:
\begin{itemize}
\item {} 
No se seleccionó un medicamento.

\item {} 
No se ingresó una cantidad.

\item {} 
La cantidad ingresada no posee un formato correcto.

\item {} 
La cantidad ingresada es menor a cero.

\item {} 
La cantidad ingresada supera el stock disponible.

\end{itemize}

Una vez completado el formulario, el usuario deberá presionar el botón “Guardar” y el sistema se encargara de agregar el nuevo detalle al pedido.
El usuario podrá seguir dando de alta nuevos detalles, hasta donde considere necesario. Una vez que esto suceda deberá presionar el botón “Cerrar” y la ventana emergente desaparecerá.


\subsubsection{Baja Detalle}
\label{pedidosclinica:baja-detalle}
Si el usuario desea eliminar un detalle del pedido de clínica, deberá seleccionar el detalle que desea eliminar y presionar el botón de “Baja detalle”.

CAPTURA BOTÓN

Una vez realizado el paso anterior aparecerá la siguiente ventana emergente (modal):

CAPTURA

En esta parte el usuario deberá decidir si confirma la eliminación del detalle o no. Si desea confirmar la eliminación deberá presionar el botón “Confirmar”, caso contrario, presionará el botón “Cancelar”.


\subsubsection{Modificar Detalle}
\label{pedidosclinica:modificar-detalle}
Si el usuario desea modificar un detalle del pedido de clínica, deberá seleccionar el detalle que desea actualizar y presionar el botón de “Modificar detalle”.

CAPTURA

Una vez realizado el paso anterior aparecerá la siguiente ventana emergente (modal):

CAPTURA

En esta parte el usuario se le presentará un formulario con la información actual del detalle y deberá modificar la información que considere necesaria.
El sistema siempre validará que la información ingresada sea correcta. En caso de que los datos ingresados sean incorrectos el sistema lo informará.
En este punto, las posibles causas de errores son:
\begin{itemize}
\item {} 
No se ingresó una cantidad.

\item {} 
La cantidad ingresada no posee un formato correcto.

\item {} 
La cantidad ingresada es menor a cero.

\item {} 
La cantidad ingresada supera el stock disponible.

\end{itemize}

Una vez completado el formulario, el usuario deberá presionar el botón “Guardar” y el sistema se encargara de actualizar la información del detalle seleccionado.


\subsubsection{Registrar Pedido}
\label{pedidosclinica:registrar-pedido}
Si el usuario desea registrar el pedido de clínica, deberá presionar el botón “Registrar”.

CAPTURA

El sistema siempre validará que la información del pedido de clínica sea correcta. En caso de que esta información sea incorrecta el sistema lo informará.
En este punto, las posibles causas de errores son:
\begin{itemize}
\item {} 
El pedido no contiene detalles

\item {} 
El pedido ya ha sido registrado anteriormente

\end{itemize}

Una vez presionado el botón “Registrar”, el sistema se encargará de crear el pedido de clínica y se mostrará la siguiente ventana emergente (modal).

CAPTURA


\subsection{Formulario de Búsqueda}
\label{pedidosclinica:formulario-de-busqueda}
Si el usuario desea visualizar solo aquellos pedidos de clínica que cumplan determinados criterios, deberá utilizar el formulario de búsqueda.

CAPTURA

Este formulario cuenta con dos modalidades:
\begin{itemize}
\item {} 
Búsqueda simple: permite buscar los pedidos de clínica por clínica.

\item {} 
Búsqueda avanzada: permite buscar los pedidos de clínica por clínica y/o obra social, fecha desde y/o fecha hasta.

\end{itemize}

Todos los campos son opcionales, de no especificarse ningún criterio de búsqueda el sistema mostrará todos los pedidos de clínica.


\subsection{Ver detalles del Pedido}
\label{pedidosclinica:ver-detalles-del-pedido}
Si el usuario desea ver los detalles de un pedido, deberá seleccionar el botón de “Acción” asociado al pedido de clínica y presionar la pestaña “Ver detalles”.

CAPTURA

Una vez realizado el paso anterior aparecerá la siguiente ventana emergente (modal):

CAPTURA

Esta ventana mostrará todos los detalles que estén asociados al Pedido de clínica.


\subsection{Ver Remitos del Pedido}
\label{pedidosclinica:ver-remitos-del-pedido}
Si el usuario desea ver los remitos asociados a un pedido, deberá seleccionar el botón de “Acción” asociado al Pedido de clínica y presionar la pestaña “Ver Remitos”.

CAPTURA

Una vez realizado el paso anterior aparecerá la siguiente ventana emergente (modal):

CAPTURA

Esta ventana mostrará todos los remitos  que estén asociados al pedido de clínica.

En caso de que el pedido no tenga remitos asociados el sistema lo informará.

Si se desea generar el remito en un pdf, el usuario deberá seleccionar el botón asociado al remito correspondiente y el sistema se encargará de generar el mismo.

CAPTURA


\section{Pedidos a Laboratorio}
\label{pedidosalab:pedidos-a-laboratorio}\label{pedidosalab::doc}
Se presentará una pantalla que contendrá un listado con todos los pedidos a laboratorio que se encuentren registrados en el sistema hasta la fecha. Junto con el listado, se presentarán un conjunto de funcionalidades que permitirán manipular estos pedidos.

CAPTURA

Estas funcionalidades son:
\begin{itemize}
\item {} 
Alta de Pedido

\item {} 
Ver Detalles del Pedido

\item {} 
Ver Remitos del Pedido

\item {} 
Cancelar Pedido

\item {} 
Formulario de Búsqueda

\end{itemize}


\subsection{Alta de Pedido}
\label{pedidosalab:alta-de-pedido}
Si el usuario desea crear un nuevo pedido a laboratorio, deberá presionar el botón “Alta”. Una vez presionado este botón el sistema lo redirigirá a la siguiente pantalla.

CAPTURA

En este punto el usuario deberá seleccionar el laboratorio al cual desea realizarle el pedido y luego presionar el botón “Continuar”.

El sistema siempre validará que la información ingresada sea correcta. En caso de que los datos ingresados sean incorrectos el sistema lo informará.
En este punto, las posibles causas de errores son:
\begin{itemize}
\item {} 
No se seleccionó un laboratorio.

\end{itemize}

Una vez presionado el botón “Continuar”, se mostrará la siguiente pantalla:

CAPTURA

Esta pantalla es la encargada de visualizar los detalles que se correspondan con el pedido a laboratorio.
De forma automática el sistema se encargara de buscar y agregar al pedido aquellos detalles de pedidos de farmacia que cumplan las siguientes condiciones:
\begin{itemize}
\item {} 
Que contengan un medicamento producido por el laboratorio al cual se le está realizando el pedido.

\item {} 
Que haya stock suficiente para satisfacer el medicamento del detalle.

\item {} 
Que no se encuentren dentro de algún otro pedido a laboratorio.

\end{itemize}

Esta pantalla ofrece las siguientes funcionalidades para manipular el pedido a laboratorio:
\begin{itemize}
\item {} 
Alta Detalle.

\item {} 
Modificación Detalle.

\item {} 
Baja Detalle.

\item {} 
Registrar Pedido.

\end{itemize}


\subsubsection{Alta Detalle}
\label{pedidosalab:alta-detalle}
Si el usuario desea agregar un detalle al pedido a laboratorio, deberá presionar el botón de “Alta detalle”.

CAPTURA BOTÓN

Una vez realizado el paso anterior aparecerá la siguiente ventana emergente (modal):

CAPTURA

En esta parte el usuario se le presentará un formulario y deberá ingresar la información solicitada para dar de alta un nuevo detalle.
El sistema siempre validará que la información ingresada sea correcta. En caso de que los datos ingresados sean incorrectos el sistema lo informará.
En este punto, las posibles causas de errores son:
\begin{itemize}
\item {} 
No se seleccionó un medicamento.

\item {} 
No se ingresó una cantidad.

\item {} 
La cantidad ingresada no posee un formato correcto.

\item {} 
La cantidad ingresada es menor a cero.

\end{itemize}

Una vez completado el formulario, el usuario deberá presionar el botón “Guardar” y el sistema se encargara de agregar el nuevo detalle al pedido.
El usuario podrá seguir dando de alta nuevos detalles, hasta donde considere necesario. Una vez que esto suceda deberá presionar el botón “Cerrar” y la ventana emergente desaparecerá.


\subsubsection{Baja Detalle}
\label{pedidosalab:baja-detalle}
Si el usuario desea eliminar un detalle del pedido a laboratorio, deberá seleccionar el detalle que desea eliminar y presionar el botón de “Baja detalle”.

CAPTURA BOTÓN

Una vez realizado el paso anterior aparecerá la siguiente ventana emergente (modal):

CAPTURA

En esta parte el usuario deberá decidir si confirma la eliminación del detalle o no. Si desea confirmar la eliminación deberá presionar el botón “Confirmar”, caso contrario, presionará el botón “Cancelar”.


\subsubsection{Modificar Detalle}
\label{pedidosalab:modificar-detalle}
Si el usuario desea modificar un detalle del pedido a laboratorio, deberá seleccionar el detalle que desea actualizar y presionar el botón de “Modificar detalle”. Solo se podrán actualizar aquellos detalles que no se correspondan con pedidos de farmacia, es decir, aquellos que el sistema agrega automáticamente al ingresar a esta pantalla.

CAPTURA

Una vez realizado el paso anterior aparecerá la siguiente ventana emergente (modal):

CAPTURA

En esta parte al usuario se le presentará un formulario con la información actual del detalle y deberá modificar la información que considere necesaria.
El sistema siempre validará que la información ingresada sea correcta. En caso de que los datos ingresados sean incorrectos el sistema lo informará.
En este punto, las posibles causas de errores son:
\begin{itemize}
\item {} 
No se ingresó una cantidad.

\item {} 
La cantidad ingresada no posee un formato correcto.

\item {} 
La cantidad ingresada es menor a cero.

\end{itemize}

Una vez completado el formulario, el usuario deberá presionar el botón “Guardar” y el sistema se encargara de actualizar la información del detalle seleccionado.


\subsubsection{Registrar Pedido}
\label{pedidosalab:registrar-pedido}
Si el usuario desea registrar el pedido a laboratorio, deberá presionar el botón “Registrar”.

CAPTURA

El sistema siempre validará que la información del pedido a laboratorio sea correcta. En caso de que esta información sea incorrecta el sistema lo informará.
En este punto, las posibles causas de errores son:
\begin{itemize}
\item {} 
El pedido no contiene detalles

\item {} 
El pedido ya ha sido registrado anteriormente

\end{itemize}

Una vez presionado el botón “Registrar”, el sistema se encargará de crear el pedido a laboratorio  y se mostrará la siguiente ventana emergente (modal).

CAPTURA


\subsection{Formulario de Búsqueda}
\label{pedidosalab:formulario-de-busqueda}
Si el usuario desea visualizar solo aquellos pedidos a laboratorio que cumplan determinados criterios, deberá utilizar el formulario de búsqueda.

CAPTURA

Este formulario cuenta con dos modalidades:
\begin{itemize}
\item {} 
Búsqueda simple: permite buscar los pedidos a laboratorio por laboratorio.

\item {} 
Búsqueda avanzada: permite buscar los pedidos a laboratorio por laboratorio y/o fecha desde y/o fecha hasta.

\end{itemize}

Todos los campos son opcionales, de no especificarse ningún criterio de búsqueda el sistema mostrará todos los pedidos a laboratorio.


\subsection{Cancelar un Pedido}
\label{pedidosalab:cancelar-un-pedido}
Si el usuario desea cancelar un pedido, deberá seleccionar el botón de “Acción” asociado al pedido a laboratorio y presionar la pestaña “Cancelar”. Solo se podrán cancelar aquellos pedidos a laboratorio que se encuentren en un estado “Pendiente”.

CAPTURA

Una vez realizado el paso anterior aparecerá la siguiente ventana emergente (modal):

CAPTURA

En esta parte el usuario deberá decidir si confirma la eliminación del pedido a laboratorio. Si desea confirmar la eliminación deberá presionar el botón “Confirmar”, caso contrario, presionará el botón “Cancelar”.


\subsection{Ver detalles del Pedido}
\label{pedidosalab:ver-detalles-del-pedido}
Si el usuario desea ver los detalles de un pedido, deberá seleccionar el botón de “Acción” asociado al pedido a laboratorio y presionar la pestaña “Ver detalles”.

CAPTURA

Una vez realizado el paso anterior aparecerá la siguiente ventana emergente (modal):

CAPTURA

Esta ventana mostrará todos los detalles que estén asociados al pedido a laboratorio seleccionado.


\subsection{Ver Remitos del Pedido}
\label{pedidosalab:ver-remitos-del-pedido}
Si el usuario desea ver los remitos asociados a un pedido, deberá seleccionar el botón de “Acción” asociado al pedido a laboratorio y presionar la pestaña “Ver Remitos”.

CAPTURA

Una vez realizado el paso anterior aparecerá la siguiente ventana emergente (modal):

CAPTURA

Esta ventana mostrará todos los remitos  que estén asociados al pedido a laboratorio seleccionado.
En caso de que el pedido no tenga remitos asociados el sistema lo informará.

Si se desea generar el remito en un pdf, el usuario deberá seleccionar el botón asociado al remito correspondiente y el sistema se encargará de generar el mismo.

CAPTURA


\section{Recepcion de Pedidos de Laboratorio}
\label{receppedidosdelab::doc}\label{receppedidosdelab:recepcion-de-pedidos-de-laboratorio}

\section{Registrar Devolución de Medicamentos Vencidos}
\label{devolucionvencidos::doc}\label{devolucionvencidos:registrar-devolucion-de-medicamentos-vencidos}
Se presentará una pantalla en la cual el usuario deberá seleccionar el laboratorio sobre el cual desea realizar una devolución de medicamentos vencidos, y luego presionar el botón “Confirmar”.

CAPTURA

En caso de que no existan laboratorios con medicamentos vencidos, el selector no mostrara opciones.

El sistema siempre validará que la información ingresada sea correcta. En caso de que los datos ingresados sean incorrectos el sistema lo informará. En este punto, las posibles causas de errores son:
\begin{itemize}
\item {} 
No se ingresó un laboratorio.

\end{itemize}

Una vez realizado el paso anterior el usuario sera redirigido a la siguiente pantalla:

CAPTURA


\subsection{Registrar Devolución}
\label{devolucionvencidos:registrar-devolucion}
Esta pantalla es la encargada de visualizar los lotes de medicamentos vencidos del laboratorio seleccionado.

La única funcionalidad de esta pantalla es registrar la devolución de los lotes de medicamentos listados.

Si el usuario desea registrar la devolución de medicamentos vencidos, deberá presionar el botón “Registrar”.

CAPTURA

Una vez presionado el botón “Registrar”, el sistema mostrará la siguiente ventana emergente (modal).

CAPTURA


\section{Registrar Recepcion de Medicamentos para Reemplazar Vencidos}
\label{reemplazovencidos:registrar-recepcion-de-medicamentos-para-reemplazar-vencidos}\label{reemplazovencidos::doc}

\chapter{Perfil}
\label{perfil:perfil}\label{perfil::doc}

\section{Agregar Usuario}
\label{agregarusuario::doc}\label{agregarusuario:agregar-usuario}


\renewcommand{\indexname}{Índice}
\printindex
\end{document}
